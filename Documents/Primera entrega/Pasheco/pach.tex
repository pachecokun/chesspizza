\documentclass[oneside,10pt]{book}


\usepackage{cdtBook}
\usepackage{usecases}
\usepackage{gensymb}

\begin{document}
\chapter{Modelo de la Interacción}

\subsection{IU1 Página principal}

\subsubsection{Objetivo}
Mostrar información general a los usuarios y dar acceso a las funciones del sistema.

\subsubsection{Diseño}
Esta pantalla aparece al ingresar al sitio desde la URL pizza.escom.xyz.

\begin{figure}[htbp!]
	\centering
	\includegraphics[width=1\textwidth]{img/principal}
	\caption{IU1 Página principal}
\end{figure}


\subsubsection{Salidas}
\begin{itemize} 
	\item Imágenes representativas de la cadena
	\item Menú de navegación
\end{itemize}
\subsubsection{Entradas}
\begin{itemize}
	\item Ninguna
\end{itemize}

\subsubsection{Comandos}
\begin{itemize}
	\item \IUbutton{INICIO}:  Muestra la interfaz \IUref{UI1}{Página principal}.	
	\item \IUbutton{ORDENAR}:  Muestra la interfaz \IUref{UI3}{Ordenar pizza}.	
	\item \IUbutton{CONSULTAR ORDEN}:  Muestra la interfaz \IUref{UI4}{Consultar orden(cliente)}.	
	\item \IUbutton{SUCURSALES}:  Muestra la interfaz \IUref{UI}{}.	
	\item \IUbutton{INICIAR SESIÓN}:  Muestra la interfaz \IUref{UI2}{Iniciar sesión}.	
\end{itemize}

\subsubsection{Mensajes}
Ninguno

\subsection{IU8 Consultar órdenes(chef)}

\subsubsection{Objetivo}
Que los chefs obtengan la información de las órdenes que deben preparar

\subsubsection{Diseño}
Esta pantalla aparece al iniciar sesión como chef en el sistema.

\begin{figure}[htbp!]
	\centering
	\includegraphics[width=1\textwidth]{img/chef}
	\caption{IU8 Consultar órdenes(chef)}
\end{figure}


\subsubsection{Salidas}
\begin{itemize} 
	\item Lista de órdenes de la sucursal a la que pertenece el chef, ordenadas primero la más reciente, con la siguiente información:
	\begin{itemize}
		\item Número de orden
		\item Hora de orden
		\item información de cada pizza de la orden:
		\begin{itemize}
			\item Cantidad
			\item Nombre de pizza en caso de ser especial o en otro caso se le asigna el nombre "Pizza personalizada"
			\item Lista de ingredientes
		\end{itemize}
	\end{itemize}
	\item Nota: La lista se actualizará cada 15 segundos
\end{itemize}
\subsubsection{Entradas}
\begin{itemize}
	\item Identificador de sucursal
\end{itemize}

\subsubsection{Comandos}
\begin{itemize}
	\item \IUbutton{Orden lista}:  Se hace click en el botón cuando una orden ha sido preparada en su totalidad y está lista para ser repartida. Al hacer click sobre él, el estádo de la orden se modifica y es removida de la lista de ordenes por preparar.
\end{itemize}

\end{document}